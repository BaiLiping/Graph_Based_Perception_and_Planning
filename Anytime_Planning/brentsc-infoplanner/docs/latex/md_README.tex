This respository accompanies the 2018 R\+AL paper\+: \href{http://ieeexplore.ieee.org/document/8260881/}{\tt http\+://ieeexplore.\+ieee.\+org/document/8260881/}. This work considers the problem of planning trajectories for a team of mobile robots that efficiently gather information about a target process of interest. In this simulation, a number of mobile robots following a unicycle motion model with discretized motion primitives, aim to track the evolution of a number of moving targets following a double-\/integrator model driven by Gaussian noise. The algorithm is implemented in C++, and the simulation can be configured to run from the command line with C++, or with Python via Py\+Bind11 for visualization of the simulation environment.

\subsection*{Dependencies}

Ensure that the following dependencies are installed\+:


\begin{DoxyItemize}
\item Eigen3
\item Yaml-\/\+C\+PP
\item Pybind11
\item C\+G\+AL
\end{DoxyItemize}

\subsection*{Compilation}

\begin{DoxyVerb}cd <install_location>
mkdir build && cd build
cmake ..
make
cd ..
\end{DoxyVerb}


\subsection*{Example Usage in C++}

Experiments can be run using C++ or M\+A\+T\+L\+AB. In C++, the configuration for the robot starting locations, sensing model, and other parameters is done in data/init\+\_\+info\+\_\+planner\+\_\+\+S\+E2.\+yaml. C\+SV files containing the resulting data are generated in the results folder.

The code can be executed\+: \begin{DoxyVerb}build/bin/test_info_planner_SE2 data/init_info_planner_ARVI.yaml
\end{DoxyVerb}


\subsection*{Usage in Python}

Note\+: Currently plotting is only supported for single robots.

Run the following script\+: \begin{DoxyVerb}python script/python/pyInfo.py
\end{DoxyVerb}


To change the scenario file, edit the python script parameters import, i.\+e. \begin{DoxyVerb}params = IG.Parameters('data/init_info_planner_ARVI.yaml')
\end{DoxyVerb}


To point to the desired simulation configuration to run.

\subsection*{Immediate To-\/\+Do\textquotesingle{}s}


\begin{DoxyItemize}
\item Clean up the way that Coordinate Descent Planning is implemented, e.\+g. by wrapping the A\+R\+VI Planner into a new Planner that takes in a team of Robots, rather than the user having to manage the Coordinate Descent algorithm themselves.
\item Clean up the Target Model interface to neatly handle multiple targets, data association, and retrieval of belief states.
\item Fix Python Plotting code to work for Multiple Robots
\item Clean up State Space / E\+NV representation to allow for motion primitives on S\+E(3), and define distance metrics on the Pose types themselves, for cleaner usage in R\+VI / A\+R\+VI delta-\/crossing.
\item Add support for Drawing obstacle maps in 2D
\item Longer Term\+: Make Python plotting code for 3D environments..
\end{DoxyItemize}

\section*{Future Features to be supported}


\begin{DoxyItemize}
\item Robots and targets are both general \textquotesingle{}Agents\textquotesingle{}, which can have a full state space in S\+E(3). Robots can hold a belief (target) model, for any agents whose pose they wish to estimate (including itself). This model can be converted by the planner into a Linear Gaussian model, and used for target tracking.
\item The above means each state Space must be convertible to S\+E(3), and vice versa. Either S\+E(3) should be the baseline, and we project down into lower spaces for control, or the other spaces exist and we lift into S\+E(3) for generic operations on the robot state. I vote for making everything S\+E(3) and projecting into lower spaces for control.
\item Come up with a good interface for what an Agent is, and what traits it posesses. I.\+e. robot has a state Space (S\+E(3)), an action space (on some subspace of S\+E(3)), one or more sensors, a planner, which generates a trajectory, and a controller to follow the trajectory.
\item We should include drivers for the sensors in the Sensor package, which interface between hardware and the Sensor models we have, i.\+e. Gazebo / real L\+I\+D\+AR or camera into our camera or range and bearing sensors.
\item A R\+OS wrapper which can allow us to wrap robot topics and sensors from Gazebo into the appropriate pieces of our module.
\item Software should be modular and split into packages that are easy to pull in, and maintained in a repository. I.\+e. Estimation, planning, Base (robots and state spaces perhaps).
\item Scenario files. The programming task could ultimately be reduced to a user writing a new scenario file. To accomplish a particular task, the user will configure the robots and sensors appropriately, then call the appropriate planner for the scenario.
\end{DoxyItemize}

\section*{R\+OS Wrapper}


\begin{DoxyItemize}
\item Create a Node that maps appropriate R\+OS topics into the inputs and outputs of the regular node. (D\+O\+NE)
\item What topics does this node subscribe to? For each robot in the team, it subscribes to a position, i.\+e. nav\+\_\+msgs/odom type. (D\+O\+NE)
\item Right now this is a centralized planner, so spawning the planner will have to remap topics for all robots into this node. (D\+O\+NE)
\item This should publish output velocity commands to some node, which will be remapped as determined by the simulation. (D\+O\+NE)
\item Other world parameters needed are the map for collision checking. The remaining parameters are planner specific, or related to the robot dynamics. (D\+O\+NE)
\item Later on, this will also subscribe to sensors in order to estimate the target. But for the very first iteration, this will be a static target with a fixed location provided by R\+OS parameters. 
\end{DoxyItemize}